
% The \section{} command formats and sets the title of this
% section. We'll deal with labels later.
\section{Introduction}
\label{sec:intro}

Linear, polynomial, and other forms of regression analysis are commonly
used in the field of statistics in order to fit a function to a data set. Produced
functions can be used to predict behavior at arbitrary input values. Typically,
regression analysis is used upon data sets for which the approximate
function \textit{type} (e.g. linear, quadratic, etc.) is known. Traditional
regression analysis is not well-suited, however, for data sets which
approximate an unknown function type. \\

One tried solution for such scenarios is symbolic regression by way of
a genetic algorithm --- an algorithm that generates sets of candidate
solutions, randomly splices together traits from pairs of fit "parents"
into "children" over successive generations, and occasionally mutates
pieces of those candidates, until a strongly fit solution is found. \\

In our paper we introduce expression trees as a means of representing
functional expressions in such a way that they can be easily spliced by
a genetic algorithm. We discuss our own implementation of symbolic
regression for two known data sets representing unknown functions,
inspired by the work of John Koza. \cite{examplerun} \\

todo: include expression tree graphic from http://www.genetic-programming.com/gpquadraticexample.html and cite

