
\section{Experiments}
\label{sec:expts}

This experiment was designed to use a genetic algorithm to adapt a population of expression trees to correctly identify two unknown functions.  In separate runs, populations of 1000 expression trees were adapted through 50 generations, with the goal of matching the goal functions.  \\

Genetic crossover was achieved by probabilistically combining expression trees based on their fitness functions.  The fitness score of each tree is calculated as the reciprocal of the total error between the expression tree and goal function, evaluated at a large number of points.  The reciprocal is taken in order to give more accurate trees, with lower sum errors, higher fitness scores.  Trees were then chosen randomly according to a probabilistic distribution weighted by the fitness scores, with trees replaced back into the population after each crossover.\\

One of the main concepts behind genetic algorithms is that parents in the algorithm pass on traits recognizably to children.  By choosing the most fit expression trees to reproduce, the algorithm should, after a number of generations, converge on a locally optimal solution. In order for preferable characteristics from parent expression trees to be transferred to child trees, the crossover operation chose a random node from each tree, then swapped the subtrees extending from the chosen nodes.  Child trees therefore were composed of partial sections of each of the parent trees.  Because trees were chosen with replacement, and combine randomly during each crossover, populations may have multiple identical trees or trees made by combining the same two parent trees at different nodes. \\

After each crossover operation, a portion of the population was mutated in order to introduce random variation. For a given expression tree, the mutation operation randomly selected a node and altered the value or mathematical operation at that node based on the original value.  To avoid damaging the behavior of the tree too much, thereby potentially lowering the accuracy of the algorithm, the mutated operation or value was chosen to structurally match the original node.  Leaf nodes were only replaced with terminals: randomly chosen constants or appropriate variables.  Nodes originally containing binary operations ($+$, $-$, $\times$, $/$) were only replaced with other operations of that nature. \\

It is of note that none of the nodes of the expression trees contain the power operation, e.g. $x^2$.  This was a code optimization choice made because power operations can be created through multiplication of appropriate variables.  Removing power operations decreased the chance that an expression tree would become arbitrarily complicated through high powers, instead favoring low integer powers -- $x$, $x^2$, $x^3$. \\

%In this section, you should describe your experimental setup. What
%were the questions you were trying to answer? What was the
%experimental setup (number of trials, parameter settings, etc.)? What
%were you measuring? You should justify these choices when
%necessary. The accepted wisdom is that there should be enough detail
%in this section that I could reproduce your work \emph{exactly} if I
%were so motivated.
